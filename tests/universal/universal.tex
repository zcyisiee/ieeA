%!TEX TS-program = xelatex
\documentclass{article}
\usepackage{amsmath}
\usepackage{amssymb}
\usepackage{graphicx}
\usepackage{hyperref}
\usepackage{listings}
\usepackage{algorithm}
\usepackage{algorithmic}
\usepackage{tikz}
\usepackage{fancyvrb}
\usepackage{xparse}
\usepackage{xcolor}

\renewcommand{\includegraphics}[2][]{\fbox{#2}}

\NewDocumentEnvironment{minted}{O{} m}
{\VerbatimEnvironment\begin{Verbatim}}
{\end{Verbatim}}

\title{Universal Parser Test: $E=mc^2$ and \textbf{Bold}}
\author{Author One \and Author Two}

\begin{document}
\maketitle

\begin{abstract}
This abstract contains text for chunking, a citation \cite{key1}, a label \label{abs:core}, and inline math $a^2 + b^2 = c^2$.
\end{abstract}

\section{Intro and Escapes}\label{sec:intro}
This paragraph includes escaped characters: \{ \} \$ \% \& \# \_ \^{} \~{} and a URL \url{https://example.com/path?x=1&y=2}. It is long enough to become a chunk.

Unicode sample: naive cafe, resume, and accented characters like café, naïve, and résumé for UTF-8 coverage.

\paragraph{A short paragraph}
More text in a paragraph with \textbf{bold}, \emph{emphasis}, and a footnote\footnote{Footnote with a cite \cite{key2} and math $x+y$.}.

\subsection{Lists}
\begin{itemize}
\item First item references Section~\ref{sec:intro} and uses \textit{formatting}.
\item Second item includes \href{https://example.com}{a link} and math $x_1$.
\end{itemize}

\begin{enumerate}
\item Enumerated item with $f(x)=x^2$ and \label{itm:first}.
\end{enumerate}

\section{Math and Environments}\label{sec:math}
Inline math $x + y$ and \(z = 1\) appear here.
Display math:
\[
x = y + z
\]
\begin{equation}
E = mc^2
\end{equation}
\begin{align}
a &= b \\
c &= d
\end{align}
\begin{gather}
u = v \\
w = z
\end{gather}
\begin{multline}
x + y + z + t + u + v \\
+ w + r
\end{multline}
\begin{equation}
\begin{split}
p &= q + r \\
s &= t
\end{split}
\end{equation}

\section{Figures and Tables}
\begin{figure}
\centering
\includegraphics[width=0.4\textwidth]{example-image}
\caption{Figure caption with math $k^2$ and \textbf{bold} text.}
\label{fig:sample}
\end{figure}

\begin{table}
\centering
\caption{Table caption with a reference to Fig.~\ref{fig:sample}.}
\label{tab:sample}
\begin{tabular}{|l|c|r|}
\hline
Left & Center & Right \\
\hline
A & B & C \\
1 & 2 & 3 \\
\hline
\end{tabular}
\end{table}

\section{Protected Blocks}
\begin{verbatim}
$not math$ \textbf{not bold} % not a comment here
\end{verbatim}

\begin{lstlisting}[language=Python]
def f(x):
    return x**2  # $not math$
\end{lstlisting}

\begin{minted}[linenos]{python}
print("hello $world$")
\end{minted}

\begin{algorithm}
\begin{algorithmic}
\STATE Initialize $x \leftarrow 0$
\STATE Update $x \leftarrow x + 1$
\end{algorithmic}
\end{algorithm}

\begin{tikzpicture}
\draw (0,0) -- (1,1);
\node at (0.5,0.5) {$x$};
\end{tikzpicture}

\section{Input Include}
This line precedes an input.
%!TEX root = ../universal.tex
This content comes from extra.tex and should be flattened into the main document. It contains inline math $1+1=2$ and a label \label{sec:included}.

This line follows the input.

\begin{thebibliography}{9}
\bibitem{key1} A. Author. Title of the first reference. 2024.
\bibitem{key2} B. Author. Title of the second reference. 2023.
\bibitem{key3} C. Author. Title of the third reference. 2022.
\end{thebibliography}

\end{document}
